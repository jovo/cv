\documentclass[10pt,colorlinks=true,urlcolor=blue]{moderncv}
\usepackage{todonotes}
\usepackage{ulem}
\usepackage{utopia}
\usepackage[unicode]{hyperref}
\usepackage{color}
\usepackage{pdfpages}
\usepackage{breakurl}
\usepackage[yyyymmdd]{datetime}
\moderncvtheme[blue]{classic}
\usepackage[utf8]{inputenc}
%JLP\usepackage[resetlabels]{multibib}
\usepackage[%
    backend=biber,
    defernumbers=true,
    refsection=section,
    sorting=ydmdt,
    firstinits=false,
    maxbibnames=999]{biblatex}
%\usepackage{etoolbox}
%\addbibresource{peer.bib}
%\addbibresource{pre.bib}
%\addbibresource{talks.bib}
%\addbibresource{other_talks.bib}
%\addbibresource{posters.bib}
\appto{\bibsetup}{\raggedright}
%%
\addbibresource{pubs_peer_reviewed.bib}
\addbibresource{pubs_in_review.bib}
\addbibresource{pubs_conf.bib}
\addbibresource{pubs_other.bib}
\addbibresource{pubs_tech_reports.bib}
\addbibresource{pubs_excluded_entries.bib}
\addbibresource{talks_invited.bib}
\addbibresource{talks_other.bib}
\addbibresource{talks_excluded_entries.bib}
\addbibresource[label=posters]{posters.bib}
\newcommand*{\subsubsectionfont}{\uline{\large\mdseries\itshape}}% New subsubsection font
\newcommand*{\subsubsectionstyle}[1]{{\subsubsectionfont\textcolor{color1}{#1}}}
\makeatletter
\NewDocumentCommand{\subsubsection}{sm}{%
  \par\addvspace{1ex}%
  \phantomsection{}% reset the anchor for hyperrefs
  \addcontentsline{toc}{subsubsection}{#2}%
  \begin{tabular}{@{}p{\hintscolumnwidth}@{\hspace{\separatorcolumnwidth}}p{\maincolumnwidth}@{}}%
    \raggedleft\hintstyle{} &{\strut\subsubsectionstyle{#2}}%
  \end{tabular}%
  \par\nobreak\addvspace{0.5ex}\@afterheading}% to avoid a pagebreak after the heading
\makeatother
\DeclareRefcontext{J}{labelprefix=J} %% Peer
\DeclareRefcontext{P}{labelprefix=P} %% Pre-prints
\DeclareRefcontext{I}{labelprefix=I} %% Invited Talks
\DeclareRefcontext{T}{labelprefix=T} %% Other Talks
\DeclareRefcontext{A}{labelprefix=A} %% Abstacts and Posters

%Change Journal name from being in quotes to being italicized 
\DeclareFieldFormat[article]{journaltitle}{\textit{#1}}

% Count total number of entries in each refsection
\AtDataInput{%
  \csnumgdef{entrycount:\therefsection}{%
    \csuse{entrycount:\therefsection}+1}}

% Print the labelnumber as the total number of entries in the
% current refsection, minus the actual labelnumber, plus one
\DeclareFieldFormat{labelnumber}{\mkbibdesc{#1}}    
\newrobustcmd*{\mkbibdesc}[1]{%
  \number\numexpr\csuse{entrycount:\therefsection}+1-#1\relax}

\renewcommand*{\mkbibnamegiven}[1]{%
  \ifitemannotation{highlight}
    {\textbf{#1}}
    {
        \ifitemannotation{trainee}
            {\uline{#1}}
            {#1}
    }
}

\renewcommand*{\mkbibnamefamily}[1]{%
  \ifitemannotation{highlight}
    {\textbf{#1}}
    {
        \ifitemannotation{trainee}
            {\uline{#1}}
            {#1}
    }
}

%\usepackage{pdfpages}
\usepackage{lastpage}
\usepackage{fancyhdr}
\pagestyle{fancy}
\pagenumbering{roman}
\chead{\textbf{Joshua T. Vogelstein, Ph.D.}, Assistant Professor, JHU -- Curriculum Vitae}
\rhead{\textbf{\today}}

\fancyfoot[C]{Page \thepage\ of \pageref*{LastPage}}

\DeclareSortingTemplate{ydmdt}{
  \sort{
    \field{presort}
  }
  \sort[final]{
    \field{sortkey}
  }
  \sort[direction=descending]{
    \field{sortyear}
    \field{year}
  }
  \sort[direction=descending]{
    \field[padside=left,padwidth=2,padchar=0]{month}
    \literal{00}
  }
  \sort[direction=descending]{
    \field[padside=left,padwidth=2,padchar=0]{day}
    \literal{00}
  }
  \sort{
    \field[padside=left,padwidth=4,padchar=0]{volume}
    \literal{0000}
  }
  \sort{
    \field{sorttitle}
  }
}

\usepackage[scale=0.8]{geometry}
\newcommand{\cvdoublecolumn}[2]{%
  \cvline{}{}{%
    \begin{minipage}[t]{\listdoubleitemmaincolumnwidth}#1\end{minipage}%
    \hfill%
    \begin{minipage}[t]{\listdoubleitemmaincolumnwidth}#2\end{minipage}%
    }%
}
%
% usage: \cvreference{name}{address line 1}{address line 2}{address line 3}{address line 4}{e-mail address}{phone number}
% Everything but the name is optxional
% If \addresssymbol, \emailsymbol or \phonesymbol are specified, they will be used.
% (Per default, \addresssymbol isn't specified, the other two are specified.)
% If you don't like the symbols, remove them from the following code, including the tilde ~ (space).

\newcommand{\cvreference}[7]{%
    \textbf{#1}\newline% Name
    \ifthenelse{\equal{#2}{}}{}{\addresssymbol~#2\newline}%
    \ifthenelse{\equal{#3}{}}{}{#3\newline}%
    \ifthenelse{\equal{#4}{}}{}{#4\newline}%
    \ifthenelse{\equal{#5}{}}{}{#5\newline}%
    \ifthenelse{\equal{#6}{}}{}{\texttt{#6}\newline}%
    \ifthenelse{\equal{#7}{}}{}{\phonesymbol~#7}}

  \AtBeginDocument{\recomputelengths}
  \firstname{Joshua T.~}
  \familyname{Vogelstein}
  \address{Assistant Professor, \\
   Department of Biomedical Engineering \\
   Johns Hopkins University \\
   }{}
  \email{jovo@jhu.edu}
  \homepage{jovo.me}


\begin{document}
\pagecolor{white}
\maketitle


\section{Personal Information}
\subsection{Primary Appointment}
\cventry{08/14 -- }{Assistant Professor}{Department of Biomedical Engineering}{JHU, Baltimore, MD, USA}{}{}

\subsection{Joint Appointments}
\cventry{09/19 -- }{Joint Appointment}{Department of Biostatistics}{Johns Hopkins University, Baltimore, MD, USA}{}{}
\cventry{08/15 -- }{Joint Appointment}{Department of Applied Mathematics and Statistics}{JHU, Baltimore, MD, USA}{}{}
\cventry{08/14 -- }{Joint Appointment}{Department of Neuroscience}{JHU, Baltimore, MD, USA}{}{}
\cventry{08/14 -- }{Joint Appointment}{Department of Computer Science}{JHU, Baltimore, MD, USA}{}{}


\subsection{Institutional and Center Appointments}
\cventry{08/15 -- }{Steering Committee}{Kavli Neuroscience Discovery Institute (KNDI)}{Baltimore, MD, USA}{}{}
\cventry{08/14 -- }{Core Faculty} {Institute for Computational Medicine}{JHU, Baltimore, MD, USA}{}{}
\cventry{08/14 -- }{Core Faculty} {Center for Imaging Science}{JHU, Baltimore, MD, USA}{}{}
\cventry{08/14 -- }{Assistant Research Faculty}{Human Language Technology Center of Excellence}{JHU, Baltimore, MD, USA}{}{}
\cventry{10/12 -- }{Affiliated Faculty}{Institute for Data Intensive Engineering and Sciences}{JHU, Baltimore, MD, USA}{}{}


\subsection{Education \& Training}
\cventry{2003 -- 2009}{Ph.D in Neuroscience}{Johns Hopkins School of Medicine}{\newline Advisor: Eric Young}{\newline \textbf{Thesis:} OOPSI: a family of optical spike inference algorithms for inferring neural connectivity from population calcium imaging }{}{}
\cventry{2009 -- 2009}{M.S. in Applied Mathematics \& Statistics}{ Johns Hopkins University}{}{}{}
\cventry{1998 -- 2002}{B.A. in Biomedical Engineering}{Washington University, St.~Louis}{}{}{}

\subsection{Academic Experience}
\cventry{08/18 -- }{\href{https://www.bme.jhu.edu/graduate/mse/degree-requirements/biomedical-data-science/}{Director of Biomedical Data Science Focus Area}}{Department of Biomedical Engineering}{Johns Hopkins University, Baltimore, MD, USA}{}{}{}
\cventry{05/16 -- }{Visiting Scientist} {Howard Hughes Medical Institute}{Janelia Research Campus, Ashburn, VA, USA}{}{}
\cventry{10/12 -- 08/14}{Endeavor Scientist}{Child Mind Institute}{New York, NY, USA}{}{}
\cventry{08/12 -- 08/14}{Affiliated Faculty}{Kenan Institute for Ethics}{Duke University, Durham, NC, USA}{}{}
\cventry{08/12 -- 08/14}{Adjunct Faculty}{Department of Computer Science}{JHU, Baltimore, MD, USA}{}{}
\cventry{12/09 -- 01/11}{Post-Doctoral Fellow}{Department of Applied Mathematics and Statistics}{Supervised by Carey E.~Priebe}{JHU, Baltimore, MD, USA}{\textbf{Research} Statistics of populations of networks.}
\cventry{06/01 -- 09/01}{Research Assistant}{Prof. Randy O'Reilly, Dept.~of Psychology}{University of Colorado, Denver, CO, USA}{}{}
\cventry{06/00 -- 09/00}{Clinical Engineer}{Johns Hopkins Hospital}{JHU, Baltimore, MD, USA}{}{}
\cventry{06/99 -- 08/99}{Research Assistant under Dr. Jeffrey Williams}{Dept. of Neurosurgery, Johns Hopkins Hospital}{Baltimore, MD, USA}{}{}
\cventry{06/98 -- 08/98}{Research Assistant under Professor Kathy Cho}{Dept. of Pathology, Johns Hopkins School of Medicine}{Baltimore, MD, USA}{}{}


\section{\href{https://neurodata.io/research/\#peer_reviewed}{Published Peer-Reviewed Research Articles}}
\begin{refsection}[pubs_peer_reviewed.bib] 
    %\newrefcontext{J}
    \nocite{*}
    \defbibnote{a1}{{Note: CV author in bold; Trainees are underlined, \\
            \textbf{(73 papers; top 10 cited 3,922 times; H-index 36; 12 first, 13 last, 48 middle authorships)} as of \today}}
    \printbibliography[%
        prenote=a1,% 
        heading=none%
        ]
    \end{refsection}

    \section{\href{https://neurodata.io/research/\#pre_prints}{Manuscripts Not Yet Accepted}}
\begin{refsection}[pubs_in_review.bib]
    %\newrefcontext{P}
    \nocite{*}
    \printbibliography[%
        title=\href{https://neurodata.io/research/\#pre_prints}{In Preparation (Peer-Review Research Articles)},%
        heading=none%
        ]
    \end{refsection}


\section{\href{https://neurodata.io/publications/\#conf}{Conference Papers}}
\begin{refsection}[pubs_conf.bib]
    \nocite{*}
    \printbibliography[%
        title=\href{https://neurodata.io/publications/\#conf}{Peer-Reviewed Conference Proceedings},%
        heading=none%
        ]
    \end{refsection}

\section{Book Chapters}
\begin{refsection}[pubs_books.bib]
    \nocite{*}
    \printbibliography[%
        title={Book Chapters},%
        heading=none%
        ]
    \end{refsection}


\section{\href{https://neurodata.io/research/\#pre_prints}{Technical Reports}}
\begin{refsection}[pubs_tech_reports.bib] %, pubs_excluded_entries.bib]
    %\newrefcontext{P}
\nocite{*}
\printbibliography[%
    title=\href{https://neurodata.io/publications/\#pre_prints}{Technical Reports},%
    heading=none%
    ]
\end{refsection}

\section{\href{https://neurodata.io/research/\#other}{Other Publications}}
\begin{refsection}[pubs_other.bib] %, pubs_excluded_entries.bib]
    %\newrefcontext{P}
    \nocite{*}
    \printbibliography[%
        title=\href{https://neurodata.io/publications/\#other}{Other Publications},%
        heading=none%
        ]
    \end{refsection}
    


    \section{Funding}
    The table below shows my direct (total) cost expenditures since being hired, indicating a steady increase >30\% per year.\newline \newline
    
    \centering
    \begin{tabular}{p{3cm}p{2cm}p{3cm}}
    Financial Year & Direct & Total \\
    2015: & \$113,761 & \$168,924 \\
    2016: & \$360,123 & \$524,225 \\
    2017: & \$459,523 & \$709,019 \\
    2018: & \$550,011 & \$887,186 \\
    2019: & \$850,836 & \$1,366,308
    \end{tabular}
    
    \subsection{External Research Support: Current}
    
    \cventry{2020 - 2025}{NSF}{AI Institute: Planning: BI4ALL: Understanding Biological}{NSF 20-503}
    {\newline PI: Kording, K.
    \newline Role on Project: Co-Investigator
    \newline Term: 01-Oct-2020 to 31-Jul-2022
    \newline Funding to lab, current year: \$79,629
    \newline The goal of this project is to plan an AI institution via several meetings and workshops}{}{}{}
    
    \cventry{2020-2025}{National Science Foundation}{Collaborative Research: Transferable, Hierarchical, Expensive, Optimal, Robust, Interpretable Networks}{NSF 20-540}
    {\newline PI: Vidal, R.
    \newline Role on Project: Co-Investigator
    \newline Term: 01-Sep-2020 to 31-Aug-2025
    \newline Funding to lab, entire period: \$1,650,000 (direct)
    \newline Funding to lab, current year: \$660,000 (direct)
    \newline The goal of this project is to develop a mathematical, statistical and computational framework that helps explain the success of current network arcitectures, understand its pitfalls, and guide the
    design of novel architectures with guaranteed confidence, robustness, interpretability, optimality, and transferability.}{}{}{}
    
    \cventry{2020 - 2025}{National Science Foundation}{NeuroNex2: Enabling Identification and Impact of Synaptic Weight in Functional Networks}{NSF 2014862}
    {\newline PI: Harris, K.
    \newline Role on Project: Co-Investigator
    \newline Term: 01-Apr-2020 to 31-Mar-2025
    \newline Funding to lab, entire period: \$609,294 (direct) \$997,719 (total)
    \newline Funding to lab, current year: \$121,587 (direct) \$199,543 (total)
    \newline The goal is to develop the requisite technology to understand the impact of synaptic weight on functional networks}{}{}{}
    
    \cventry{2020 - 2023}{National Institutes of Health}{Graspy: A python package for rigorous statistical analysis of populations of attributed connectomes}{NIH MH-19-147}
    {\newline PI: Vogelstein, J.
    \newline Term: 01-Jul-2020 to 30-Jun-2023
    \newline Funding to the lab, entire period: \$861,240 (direct) \$1,416,279 (total)
    \newline Funding to the lab, current year: \$283,301 (direct) \$471,082 (total)
    \newline The goal of this project is to establish a state-of-the-art toolbox for analysis of connectomes, spanning taxa, scale, and complexity. we will develop and extend implementations to enable neurobiologists to (1) estimate latent structure from attributed connectomes, (2) identify meaningful clusters among populations of connectomes, and (3) detect relationships between connectomes and multivariate phenotypes.}{}{}{}
    
    \cventry{2020 – 2025}{National Science Foundation}{CAREER: Foundational Statistical Theory and Methods for Analysis of Populations of Attributed}{NSF 17-537}
    {\newline PI: Vogelstein, J.
    \newline Term: 01-Jan-2020 to 31-Dec-2025
    \newline Funding to lab, entire period: \$630,230 (total) \$384,873 (direct)
    \newline Funding to lab, current year: \$126,046 (total) \$76,975 (direct)
    \newline The goal is to establish foundational theory and methods for analyzing populations of
    attributed connectomes}{}{}{}
    
    \cventry{2019 - 2023}{National Institutes of Health}{Brain Networks in Mouse Models of Aging}{NIH RO1AG066184-01}
    {\newline PI: Badea, A.
    \newline Role on Project: Co-Investigator
    \newline Term: 01-Dec-2019 to 30-Nov-2023
    \newline Funding to lab, current year: \$205,998
    \newline The goal of this grant is to generate connectomes and RNA-seq transcriptomes to characterize and differentiate APOE mice as a model of aging.
    }{}{}{}
    
    
    
    \cventry{2019 -- 2022}{National Institutes of Health}{Accessible technologies for high-throughput, whole-brain reconstructions of molecularly characterized mammalian neurons}{NIH RFA-MH-19-148}
    {\newline PI: Muller, M.
    \newline Role on Project: Co-Investigator
    \newline Term: 01-Sept-2019 to 31-Aug-2022
    \newline Funding to lab, entire period: \$1,180,445 (total) \$753,974 (direct) 
    \newline Funding to lab, current year: \$383,482 (total) \$251,325 (direct)
    \newline The overall goal of the proposal is to develop technologies for the brain wide reconstruction
    of axonal arbors of molecularly defined neurons. The proposal aims at overcoming barriers
    in neuronal labeling, imaging and computation to achieve this goal, and to develop a
    technology platform that can be scaled to all neurons of the brain}{}{}{}
    
    \cventry{2019 – 2020}{National Institutes of Health}{Reproducible imaging-based brain growth charts for psychiatry}{NIH R01MH120482-01}
    {\newline PI: Satterthwaite, T.
    \newline Role on project: Co-Investigator
    \newline Term: 01-Aug-2019 to 31-May-2020
    \newline Funding to lab, entire period: \$362,861 (total) \$231,276 (direct)
    \newline Aggregate, harmonize, and analyze existing large-scale pediatric neuroimaging datasets to
    identify normative and clinical brain growth curves}{}{}{}
    
    \cventry{2019 -- }{Microsoft}{Microsoft Research Award}{}
    {\newline PI: Vogelstein, J.
    \newline Term: Unrestricted Gift
    \newline Funding to lab, entire period: \$50,000 (total)
    \newline Research and development of neuroscience and connectomes around neuronal circuit and
    system modeling, application of time-series-of-graphs and dynamics to neuronal signaling
    analysis and connectomes, and in the abstractions of matter, math, machines that point
    toward complex systems composed of low-level components}{}{}{}{}
    
    \cventry{2018 – 2021}{National Science Foundation}{SemiSynBio: Collaborative Research: YeastOns: Neural Networks Implemented in Communication Yeast Cells}{NSF 1807369}
    {\newline PI: Schulman, E.
    \newline Role on project: Co-Investigator
    \newline Term: 16-Jul-2018 to 30-Jun-2021
    \newline Funding to lab, entire period: \$263,942 (total) \$172,971 (direct)
    \newline Funding to lab, current year: \$87,980 (total) \$57,657 (direct)
    \newline Provide neuroscience and machine learning expertise to guide the design of the
    computational learning capabilities of the system}{}{}{}
    
    \cventry{2017 – 2021}{DARPA}{Lifelong Learning Forests}{FA8650-18-2-7834}
    {\newline PI: Vogelstein, J.
    \newline Term: 01-Nov-2017 to 31-Oct-2021
    \newline Funding to lab, entire period: \$1,839,308 (total) \$1,123,474 (direct)
    \newline Funding to lab, current year: \$199,179 (total) \$121,667 (direct)
    \newline Lifelong Learning Forests (L2Fs) will learn continuously, selectively adapting to new
    environments and circumstances utilizing top-down feedback to impact low-level
    processing, with provable statistical guarantees, while maintaining computational tractability
    at scale}{}{}{}{}
    
    \cventry{2017 – 2021}{DARPA}{Continual Learning Across Synapses, Circuits, and Brain Areas}{FA8650-18-2-7834}
    {\newline PI: Tolias, A.
    \newline Role on project: Co-Investigator
    \newline Term: 01-Nov-2017 to 30-Oct-2021
    \newline Funding to lab, entire period: \$796,715 (total) \$486,666 (direct)
    \newline Funding to lab, current year: \$199,179 (total) \$121,667 (direct)
    \newline Develop the pre-processing analysis pipeline for the imaging data collected in this project}{}{}{}
    
    \cventry{2017 – 2020}{National Science Foundation}{NeuroNex Innovation Award: Towards Automatic Analysis of Multi-Terabyte Cleared Brains}{NSF 1707298}
    {\newline PI: Vogelstein, J.
    \newline Term: 01-Sept-2017 to 31-Aug-2020 (No Cost Extension)
    \newline Funding to lab, entire period: \$959,999 (total) \$588,758 (direct)
    \newline Funding to lab, current year: \$320,000 (total) \$196,252 (direct)
    \newline We propose to lower the barrier to connecting data to analyses and models by providing a
    coherent cloud computational ecosystem that minimizes current bottlenecks in the scientific
    process}{}{}{}
    
    \cventry{2017 – 2022}{National Institutes of Health}{Sensorimotor processing, decision making, and internal states: towards a realistic multiscale circuit model of the larval zebrafish brain}{NIH 1U19NS104653-01}
    {\newline PI: Engert, F.
    \newline Role on Project: Co-Investigator
    \newline Term: 01-Sept-2017 to 31-Aug-2022
    \newline Funding to lab, entire period: \$1,050,000 (total) \$655,206 (direct) (JHU sub-award)
    \newline Funding to lab, current year: \$210,000 (total) \$131,041 (direct)
    \newline Generate a realistic multiscale circuit model of the larval zebrafish’s brain – the multiscale
    virtual fish (MSVF). The model will span spatial ranges from the nanoscale at the synaptic
    level, to local microcircuits to inter-area connectivity - and its ultimate purpose is to explain
    and simulate the quantitative and qualitative nature of behavioral output across various
    timescales}{}{}{}
    
    
    \subsection{External Research Support: Completed}
    
    \cventry{2018 – 2020}{Schmidt Science Foundation}{Connectome Coding at the Synaptic Scale}{Nascent Innovation Grant 128503}
    {\newline PI: Vogelstein, J.
    \newline Term: 01-Jan-2018 to 31-Dec-2020
    \newline Funding to lab, entire period: \$250,000 (total)
    \newline Study learning and plasticity at an unprecedented scale, revealing the dynamics of large
    populations of synapses comprising an entire local cortical circuit. No previously conducted
    experiment could answer the questions about the dynamics of large populations of
    synapses, which is crucial to understanding the learning process}{}{}{}
    
    \cventry{2017 – 2020}{National Institutes of Health}{CRCNS US-German Res Prop: functional computational anatomy of the auditory cortex}{NIH 1R01DC016784-01}
    {\newline PI: Ratnanather, J.
    \newline Role on Project: Co-Investigator
    \newline Term: 01-July-2017 to 30-June-2020
    \newline Funding to lab, entire period: \$747,143 (total) \$458,519 (direct)
    \newline Create a robust computational framework for analyzing the cortical ribbon in a specific
    region: the auditory cortex}{}{}{}
    
    \cventry{2017 – 2020}{National Science Foundation}{Multiscale Generalized Correlation: A Unified Distance-Based Correlation Measure for Dependence Discovery}{NSF 1921310}
    {\newline PI: Cencheng, S.
    \newline Role on project: Co-Investigator
    \newline Term: 01-May-2017 to 30-April-2020
    \newline Funding to lab, entire period: \$200,000 (total) \$124,189 (direct)
    \newline Establish a unified methodology framework for statistical testing in high-dimensional, noisy,
    big data, through theoretical advancements, comprehensive simulations, and real data
    experiments}{}{}{}
    
    \cventry{2016 – 2020}{DARPA}{D3M: What Would Tukey Do?}{FA8750-17-2-0112}
    {\newline PI: Priebe, C.
    \newline Role on project: Co-Investigator
    \newline Term: 01-Oct-2016 to 30-Sep-2020
    \newline Funding to lab, entire period: \$4,406,360 (total) \$2,746,050 (direct)
    \newline Develop theory and methods for generating a discoverable archive of data modeling
    primitives and for automatically selecting model primitives and for composing selected primitives into complex
    modeling pipelines based on user-specified data and outcome(s) of interest}{}{}{}
    
    \cventry{2017-2019}{National Science Foundation}{NeuroNex Technology Hub: Towards the International Brain Station for Accelerating and Democratizing Neuroscience Data Analysis and Modeling}{NSF 16-569}
    {\newline PI: Vogelstein, J.
    \newline Funding to lab, entire period: \$246,773
    \newline We propose to lower the barrier to connecting data to analyses and models by providing a coherent cloud computational ecosystem that minimizes current bottlenecks in the scientific process}{}{}{}
    
    \cventry{2017 – 2018}{DARPA}{The Brain Ark}{90076467}
    {\newline PI: Vogelstein, J. 
    \newline Funding to lab, entire period: \$92,376 (total) \$56,499.08 (direct)
    \newline Characterize the statistical properties of the individual graphs, to identify circuit motifs,
    both that specialize in a species specific fashion, and that are preserved across species.
    As a test, will compare the connectomes of sea lions and coyotes}{}{}{}
    
    \cventry{2017 – 2018}{The Kavli Foundation}{The International Brain Station}{90071826}
    {\newline PI: Vogelstein, J.
    \newline Funding to lab, entire period: \$50,000 (total) \$50,000 (direct)
    \newline Take the first few steps towards building the international brain station}{}{}{}{}
    
    \cventry{2017 – 2018}{National Science Foundation}{Brain Comp Infra: EAGER: BrainLab CI: Collaborative, Community Experiments}{ACI-1649880}
    {\newline PI: Miller, B.
    \newline Role of Project: Co-Investigator
    \newline Funding to lab, entire period: \$294,599 (total) \$180,736 (direct)
    \newline The BrainLab CI prototype system will deploy an experimental-management
    infrastructure that allows users to construct community-wide experiments that implement
    data and metadata controls on the inclusion and exclusion of data}{}{}{}
    
    \cventry{2016 – 2019}{National Science Foundation}{A Scientific Planning Workshop for Coordinating Brain Research Around the Globe}{NSF 1637376 Part 1 of 2}
    {\newline PI: Vogelstein, J.
    \newline Funding to lab, entire period: \$97,950 (total) \$97,950 (direct)
    \newline This travel grant is for the expressed purposes of gathering researchers from around the
    globe to discuss the new way to further brain research during part one of a two day
    conference}{}{}{}
    
    \cventry{2016 – 2019}{National Science Foundation}{A Scientific Planning Workshop for Coordinating Brain Research Around the Globe}{NSF 1637376 Part 2 of 2}
    {\newline PI: Vogelstein, J.
    \newline Funding to lab, entire period: \$16,327 (total) \$14,491 (direct)
    \newline This travel grant is for the expressed purposes of gathering researchers from
    around the globe to further discuss advancements in brain research during the
    second part of a two day conference}{}{}{}
    
    \cventry{2015 – 2018}{DARPA}{From RAGs to Riches: Utilizing Richly Attributed Graphs to Reason from}{N66001-15-C-40401}
    {\newline PI: Vogelstein, J.
    \newline Funding to lab, entire period: \$2,103,091.60 (total) \$1,298,204 (direct)
    \newline Multiple, large, multifarious brain imaging datasets are rapidly becoming standards in
    neuroscience. Yet, we lack the tools to analyze individual datasets, much less
    populations thereof. Therefore, we will develop theory and methods to analyze and
    otherwise make such data available}{}{}{}
    
    \cventry{2014 – 2016}{DARPA}{Scalable Grain Graph Analyses Using Big-Memory, High-IPS Compute Architectures}{N66001-14-1-4028}
    {\newline PI: Burns, R.
    \newline Role on Project: Co-Investigator
    \newline Funding to lab, entire period: \$39,882 (total) \$28,272 (direct)
    \newline Build software infrastructure to enable analytics on billion node, terabyte sized networks
    using commodity hardware}{}{}{}
    
    \cventry{2014 -- 2019}{National Institutes of Health}{Synaptomes of Mouse and Man}{NIH R01NS092474}
    {\newline PI: Smith, S.
    \newline Role on project: Co-Investigator
    \newline Funding to lab, entire period: \$756,417 (total) \$491,341 (direct)
    \newline The major goals of this project are to discover the synaptic diversity and complexity in
    mammalian brains, specifically comparing and contrasting humans with mice, the leading
    experimental animal}{}{}{}
    
    \cventry{2012 – 2015}{National Institute of Biomedical Imaging and Bioengineergng}{CRCNS: Data Sharing: The EM open Connectome Project}{RO1EB16411}
    {\newline PI: Burns, R.
    \newline Role of Project: Co-Investigator
    \newline Funding to lab, entire period: \$70,823 (total) \$46,517 (direct)
    \newline Develop cyberinfrastructure to support management, visualization, storage, and analysis
    of large-scale electron microscopy data}{}{}{}{}
    



    \section{\href{https://neurodata.io/talks/}{Invited Talks}}
    \begin{refsection}[talks_invited.bib]
        %\newrefcontext{I}
        \nocite{*}
        \printbibliography[%
            title=\href{https://neurodata.io/talks/}{Invited Talks (Local)},% 
            %NB: Institutional means invited talks.  (No idea why)
            heading=none%
            ]
    \end{refsection}
        
        
    \section{\href{https://neurodata.io/talks/}{Other Talks}}
    \begin{refsection}[talks_other.bib, talks_excluded_entries.bib]
        %\newrefcontext{T}
        \nocite{*}
        \printbibliography[%
            title=\href{https://neurodata.io/talks/}{Invited Talks (International)},%
            %NB: International means non-invited talks.  (Again, no idea why)
            heading=none%
            ]
    \end{refsection}
    
    
    
    \section{\href{https://neurodata.io/posters/}{Abstracts/Poster Presentations}}
    \begin{refsection}[posters.bib]
        %\newrefcontext{T}
        \nocite{*}
        \printbibliography[%
            title=\href{https://neurodata.io/posters/}{Abstracts / Posters},%
            heading=none%
            ]
    \end{refsection}
    

\section{Educational Activities}

\subsection{Teaching Experience - Ongoing Courses}
\cventry{Fall '20}{Course Director, JHU}{EN.580.237/437/697}{\href{https://neurodatadesign.io/}{NeuroData Design I}}{enrollment 38}{}
\cventry{Spring '20}{Course Director, JHU}{EN.580.438/638}{\href{https://neurodatadesign.io/}{NeuroData Design II}}{enrollment 32}{}
\cventry{Fall '19}{Course Director, JHU}{EN.580.237/437/637}{\href{https://github.com/NeuroDataDesign/Syllabus}{NeuroData Design I}}{enrollment 46}{}
\cventry{Spring '19}{Course Director, JHU}{EN.580.438/638}{\href{https://github.com/NeuroDataDesign/Syllabus}{NeuroData Design II}}{enrollment 18}{}
\cventry{Fall '18}{Course Director, JHU}{EN.580.237/437/637}{\href{https://github.com/NeuroDataDesign/Syllabus}{NeuroData Design I}}{enrollment 22}{}
\cventry{Spring '17}{Course Director, JHU}{EN.580.238/438/638}{\href{https://github.com/NeuroDataDesign/Syllabus}{NeuroData Design II}}{enrollment 14}{}
\cventry{Winter '17}{Course Director, JHU}{EN.580.574}{BME Research Intersession}{enrollment 6}{}
\cventry{Fall '17}{Course Director, JHU}{EN.580.247/437/637}{\href{https://github.com/NeuroDataDesign/Syllabus}{NeuroData Design I}}{enrollment 15}{}
\cventry{Spring '16}{Course Director, JHU}{EN.580.468}{\href{https://github.com/Upward-Spiral-Science/Syllabus}{The Art of Data Science}}{enrollment 24}{}
\cventry{Fall '16}{Course Director, JHU}{EN.580.437}{\href{https://github.com/NeuroDataDesign/Syllabus}{NeuroData Design I}}{enrollment 16}{}
\cventry{Spring '15}{Course Director, JHU}{EN.580.694}{\href{https://github.com/openconnectome/Statistical-Connectomics-Class}{Statistical Connectomics}}{enrollment 26}{}


\subsection{Teaching Experience - One-Time}
\cventry{Spring '19}{Guest Lecturer, JHU}{EN.580.422}{Systems Bioengineering II}{}{2 Lectures}
\cventry{Spring '19}{Guest Lecturer, JHU}{AS.080.321}{Computational Neuroscience}{}{2 Lectures}
\cventry{Spring '18}{Guest Lecturer, JHU}{EN.580.422}{Systems Bioengineering II}{}{2 Lectures}
\cventry{Spring '18}{Guest Lecturer, JHU}{AS.080.321}{Computational Neuroscience}{}{2 Lectures}
\cventry{Spring '17}{Guest Lecturer, JHU}{EN.580.422}{Systems Bioengineering II}{}{2 Lectures}
\cventry{Spring '16}{Guest Lecturer, JHU}{EN.580.422}{Systems Bioengineering II}{}{2 Lectures}
\cventry{Winter '16}{Guest Lecturer, JHU}{EN.600.221}{Introduction to Connectomics}{}{1 Lecture}
\cventry{Fall '16}{Guest Lecturer, JHU}{EN.580.111}{BME Modeling and Design}{}{1 Lecture}
\cventry{Fall '15}{Course Co-Director, JHU}{Introduction to Computational Medicine}{}{}{}

\subsubsection{Educational Workshops}
\cventry{Summer '19}{\href{https://workshop.dipy.org}{DiPy Workshop}}{}{Bloomington, Indiana}{1 day  lecture on statistical connectomics}{}
\cventry{Fall '18}{\href{https://www.sfn.org/meetings/neuroscience-2018/sessions-and-events/neuroscience-2018-program}{Society for Neuroscience Annual Meeting}}{Educational Workshop}{San Diego, CA}{1 day  lecture on statistical connectomics}{}
\cventry{Fall '17}{\href{https://www.sfn.org/meetings/neuroscience-2017}{Society for Neuroscience Annual Meeting}}{Educational Workshop}{San Diego, CA}{1 day  lecture on statistical connectomics}{}
\cventry{Summer '16}{\href{http://crcns.org/previous-courses/2016_course}{CRCNS Course on Mining and Modeling of Neuroscience Data}}{Redwood Center for Theoretical Neuroscience}{University of California, Berkeley}{2 day  lecture on statistical connectomics}{}


\section{Mentorship}
\subsection{Research Track Faculty Mentorship (3)}
    \cventry{02/19 -- 12/19}{Hayden Helm, MSE}{Assistant Research Faculty}{BME}{JHU}{Lead research efforts developing theory and methods for lifelong learning}
    \cventry{08/16 -- 08/18}{Eric Perlman, PhD}{Assistant Research Scientist}{BME}{JHU}{Lead Scientist in developing storage, transfer, and visualization solutions for large data in our cloud infrastructure}
    \cventry{03/16 -- 06/20}{Jesse Patsolic, MA}{Assistant Research Faculty}{BME}{JHU}{Lead developer converting our extensions to decision forests to be merged into sklearn}


\subsection{Staff Research Scientists (5)}
    \cventry{03/20 -- }{Ali Geisa, MS}{Research Assistant}{BME}{JHU}{Researching progressive and lifelong learning theory}
    \cventry{07/19 -- 08/20}{Ronak Mehta, MSE}{Research Assistant}{BME}{JHU}{Finalizing three manuscripts on (1) uncertainty forests, (2) time-series dependence quantification, and (3) lifelong learning forests}
    \cventry{06/19 -- }{Ronan Perry}{Research Assistant}{BME}{JHU}{Developed generalized canonical correlation analysis code for analysis of high-dimensional brain imaging data in a novel meditation dataset.}
    \cventry{06/19 -- 08/20}{Devin Crowley}{Research Assistant}{BME}{JHU}{Lead developer of our scalable Python implementation of LDDMM}
    \cventry{06/18 -- 12/19}{Benjamin Falk, PhD}{Research Engineer}{BME}{JHU}{Lead software engineer, oversees all development projects, solely responsible for all cloud infrastructure}


\subsection{Postdoctoral Fellows (8)}
    \cventry{08/19 -- }{Javier How, PhD}{Postdoctoral Fellow}{Kavili NDI}{JHU}{Working on understanding how larval zebrafish generalize information from one task to perform another task, taking advantage of their optical transparency to record neuron activation.}
    \cventry{07/19 -- }{Celine Drieu, PhD}{Post-doctoral Fellow}{Kavli NDI}{JHU}{Co-Advised by Assitant Prof. Kuchibhotla, Department of Psychological and Brain Sciences. Working on understanding learning and memory using two-photon calcium imaging.}
    \cventry{07/19 -- }{Austin Grave, PhD}{Post-doctoral Fellow}{Kavli NDI}{JHU}{Co-Advised by  Prof. Richard Huganir, Department of Neuroscience. Working on understanding whole brain synaptic plasticity using genetic engineering and light microscopy imaging.}
    \cventry{08/18 -- 08/20}{Jes\'us Arroyo, PhD}{Post-doctoral Fellow}{CIS}{JHU}{Working on graph matching and joint graph embedding.}
    \cventry{07/18 -- 07/20}{Audrey Branch, PhD}{Post-doctoral Fellow}{Kavli NDI}{JHU}{Co-Advised by Prof Michela Gallagher, extending brain clearing experimental technology from mice to rats. Currently with a manuscript on biorxiv.}
    \cventry{09/16 -- 08/18}{Cencheng Shen, PhD}{Post-Doctoral Fellow}{CIS}{JHU}{Developed Multiscale Graph Correlation, which is currently the premiere hypothesis testing framework, and about to be integrated into SciPy, by far the world's leading scientific computing package. Currently an Assistent Professor in Department of Statistics at University of Delaware, and still an actice collaborator and grantee.}
    \cventry{05/16 -- 06/17}{Leo Duan, PhD}{Post-doctoral Fellow}{CIS}{JHU}{Went on to do a second postdoc with Leo Dunson (who I did my second postdoc with). Currently an Assistant Professor at University of Florida.}
    \cventry{06/16 -- 07/17}{Guilherme Franca, PhD}{Post-doctoral Fellow}{CIS}{JHU}{Worked on non-parametric clustering, with an article about to be accepted in PAMI, the leading machine learning journal.  Currently a postdoc for Rene Vidal.}

\subsection{Ph.D. Students (11)}
    \cventry{05/20 -- }{Tingshan Liu, BA}{PhD advisee}{BME}{JHU}{Currently focusing on implementation of hierarchical clustering and stochastic block modeling algorithms in Python based on gaussina mixture modeling.}
    \cventry{08/19 -- }{Michael Powell, MSE}{PhD advisee}{BME}{JHU}{Dissertation will focus on explainable artificial intelligence, spearheads collaboration with Andreas Muller, Co-Director of scikit-learn, the world's leading machine learning package.}
    \cventry{07/19 -- }{Jayanta Dey, MSE}{PhD advisee}{BME}{JHU}{Currently working on lifelong learning that aims at training a machine learning model on multiple tasks and transferring knowledge among tasks.}
    \cventry{06/19 -- }{Jaewon Chung, MSE}{PhD advisee}{BME}{JHU}{Dissertation will focus on statistics of populations of human networks. Already co-first author and middle author on multiple manuscripts.}
    \cventry{08/19 -- }{Thomas Athey, BSE}{PhD advisee}{BME}{JHU}{Dissertation will focus on MouseLight project, spearheads collaborations with Prof. Jeremias Sulam and Michael I.~Miller.}
    \cventry{08/19 -- }{Eric Bridgeford, BSE}{PhD advisee}{Department of Biostatistics}{JHU}{Dissertation will focus on statistics of human connectomes and mitigating batch effects.  Already first author on several manuscripts under review, and spearheads collaboration with Prof Brian Caffo at Biostatistics.}
    \cventry{06/19 -- }{Sambit Panda, MSE}{PhD advisee}{BME}{JHU}{Led development of Python implementation of MGC, to be integrated into SciPy.}
    \cventry{08/18 -- }{Benjamin Pedigo, BSE}{PhD advisee}{BME}{JHU}{Dissertation will focus on analysis and modeling of the world's first whole animal connectome, in collaboration with Marta Zlatic and Albert Cardona (formerly of Janelia Research Campus).  Already co-first author and middle author on multiple manuscripts.}
    \cventry{08/18 -- }{Meghana Madyastha, BSE}{PhD Co-advisee}{CS}{JHU}{Dissertation will focus on computational aspects of accelerationg learning and inference using decision forests.}
    \cventry{08/16 -- }{Vikram Chandrashekhar, BSE}{PhD advisee}{BME}{JHU}{Dissertation has focused on extending LDDMM to whole cleared brain datasets, spearheads collaboration with Prof. Karl Deisseroth's lab at Stanford, one of the world's leading neuroscientists.}
    \cventry{08/14 -- 01/18}{Tyler Tomita, PhD}{}{BME}{JHU}{Developed Sparse Projection Oblique Randomer Forest in his dissertation, currently the best performing machine learning algorithm on a standard suite of over 100 benchmark problems. Currenly a postdoc with Assistant Prof. Chris Honey of Psychology and Brain Sciences.}

\subsection{Visiting Doctoral Student }
    \cventry{03/19 -- 09/19}{Derek Pisner}{PhD advisee}{JHU/UT Austin}{}{}

\subsection{M.S. Students (10)}
    \cventry{09/20 -- }{Kaleab Alemayehu}{MS advisee}{BME}{JHU}{Working on ProgLearn to enable backward and forward knowledge transfer to improve performance on both past and future tasks using decision forests and deep networks}
    \cventry{08/20 -- }{Jong Shin}{MS advisee}{BME}{JHU}{}
    \cventry{05/20 -- }{Ali Saad-Eldin}{MS advisee}{BME}{JHU}{Working on implementing and improving combinatorial optimization algorithms, specifically the Quadratic Assignment Problem}
    \cventry{02/20 -- }{Will LeVine}{MS advisee}{BME}{JHU}{Exploring different sub-algorithms within progressive learning to alleviate harmful effects that result from training on unhelpful data}
    \cventry{09/19 -- }{Ross Lawrence}{MS advisee}{BME}{JHU}{Lead m2g developer, maintainer of neuroparc, MRI analysis lead}
    \cventry{06/19 -- 12/20}{Bijan Varjavand}{MS advisee}{BME}{JHU}{Submitted manuscript to PAMI on advancing statistics on populations of networks.}
    \cventry{10/18 -- }{Alex Loftus}{MS advisee}{BME}{JHU}{Developer of m2g, helping transition from a stand-alone package to be integrated with DiPy}
    \cventry{06/18 -- 06/19}{Drishti Mannan}{MS advisee}{BME}{JHU}{Preparing manuscript introducing novel specification for large attributed networks.}
    \cventry{06/18 -- 05/19}{Jaewon Chung}{MSE advisee}{BME}{JHU}{Co-first author of manuscript and co-lead developer of Python package for statistical analysis of networks. Currently a BME PhD student in my lab.}
    \cventry{08/14 -- 06/17}{Greg Kiar}{MSE advisee}{BME}{JHU}{Lead deveoper of m2g, the only existing ``soup to nuts'' pipeline for both functional and diffusion pipelines; co-first author of manuscript under review. Currently a PhD student at McGill University.}


\subsection{Undergraduate Students (9) (BME unless noted otherwise)}
    \cventry{09/20 -- }{Rebecca Yin}{BSE}{BME}{JHU}{Currently working on using Uncertainty Forests to generate independence tests directly}
    \cventry{08/20 -- }{Alisha Kodibagkar}{BSE}{BME}{JHU}{Assisting in the integration of brainlit packages with Azure services}
    \cventry{08/20 -- }{Shreya Singh}{BSE}{BME}{JHU}{Evaluating different methods of foreground/background classification on images of fluroescently labeled projection neurons}
    \cventry{06/19 -- }{Vivek Gopalakrishnan}{BSE}{BME}{JHU}{Winner of Pistritto Fellowship, worked on statistics of populations of connectomes in Austim mouse models and human data.}
    \cventry{06/19 -- 12/19}{Richard Guo}{BSE}{BME}{JHU}{Developed uncertainty forests, an approach for estimated posterior class probabilities, conditional entropy, and mutual information for high-dimensional data common in brain science applications.}
    \cventry{08/15 -- 08/16}{Albert Lee,BSE}{}{BME}{JHU}{Developed big data visualization tools.}
    \cventry{06/15 -- 12/15}{Ron Boger, BSE}{}{BME}{JHU}{Worked at a computational medicine start-up in Silicon Valley, worked on high-dimensional low-sample size theory.}
    \cventry{05/15 -- 05/16}{Jordan Matelsky, BSE}{}{CS and Neuroscience}{JHU}{Currently a data scientist at APL, developed a number of simple WebApps in support of big data management.}
    \cventry{02/15 -- 05/16}{Ivan Kuznetsov, BSE}{}{BME}{JHU}{Currently an MD/PhD Candidate at the UPenn, winner of \href{https://beblog.seas.upenn.edu/tag/ivan-kuznetsov/}{Soros Fellowship}, worked on analysis of data from Dr. Daniel Amen, developed matrix exploratory data analysis package.}


\subsection{High School Students (6)}
    \cventry{Summer '19}{Kareef Ullah}{Summer Intern}{BME}{JHU}{Will begin undergrad in BME at JHU in Fall 2020}
    \cventry{Summer '19}{Shunan Wu}{Summer Intern}{BME}{JHU}{Applied to BME PhD Program in Fall 2020}
    \cventry{Summer '19}{Shiyu Sun}{Summer Intern}{BME}{JHU}{Applied to BME PhD Program in Fall 2020}
    \cventry{Summer '19}{Sander Shulhoff}{Summer Intern}{BME}{JHU}{}
    \cventry{Summer '19}{Kiki Zhang}{Summer Intern}{BME}{JHU}{}
    \cventry{Summer '18}{Papa Kobina Van Dyck}{Summer Intern}{BME}{JHU}{Applied to  PhD Program in Fall 2019}

\subsection{Thesis Committee Service (BME unless noted otherwise)}
    \cventry{2019}{Browne, James}{Computer Science}{JHU Ph.D. Student}{Graduated 2019}{}
    \cventry{2019}{Mhembere, Disa}{Computer Science}{JHU Ph.D. Student}{Graduated 2019}{}
    \cventry{2018}{Kutten, Kwame}{JHU Ph.D. Student}{Graduated 2018}{}{}{}
    \cventry{2018}{Wang, Shangsi}{Applied Mathematics and Statistics}{JHU Ph.D. Student}{Graduated 2018}{}
    \cventry{2018}{Tang, Runze}{Applied Mathematics and Statistics}{JHU Ph.D. Student}{Graduated 2018}{}
    \cventry{2018}{Lee, Youjin}{Biostatistics}{JHU Ph.D. Student}{Graduated 2018}{}
    \cventry{2017}{Zheng, D}{Computer Science}{JHU Ph.D. Student}{Graduated 2017}{}
    \cventry{2017}{Binkiewicz, Norbert}{Statistics}{University of Wisconsin Ph.D. Student}{Graduated 2017}{}
    \cventry{2016}{Gray-Roncal, Will}{Computer Science}{JHU Ph.D. Student}{Graduated 2016}{}


    
\section{Service}

\subsection{Grant Review Service}
\cventry{2015}{NSF Review Panel}{Review for NSF BIG DATA Program}{}{}{}

\subsection{University Service}
\cventry{Winter '19}{Track Organizer}{AI in Healthcare: From Bench to Bedside}{Organizer for Breakout Topic Sessions on artificial intelligence}{}{}
\cventry{08/15 -- 07/18}{Co-Developer}{\href{http://icm.jhu.edu/academics/undergraduate-minor/}{Computational Medicine Minor}}{JHU, Baltimore, MD, USA}{}{}
\cventry{2015 -- 2017}{Co-Founder and Faculty Advisor}{\href{http://medhacks.org}{MedHacks}}{Medhacks is one of the first and largest hackathons dedicated specifically to hacking on medical advances, started entirely by BME undergrads at JHU}{}{}
\cventry{08/14 -- 08/18}{\href{http://icm.jhu.edu/academics/undergraduate-minor/}{Director of Undergraduate Studies}}{Institute for Computational Medicine}{JHU, Baltimore, MD, USA}{}{}


\subsection{Department Service}
\cventry{2019}{Member}{Search Committee}{BME}{Neuroengineering, 2019}{}
\cventry{2019}{Member}{Search Committee}{BME}{Data Science, 2019}{}
\cventry{2018}{Member}{Search Committee}{BME}{Neuroengineering, 2018}{}    

\subsection{Service in Scientific Community}
\cventry{2019 -- }{Mentor}{Black in AI}{}{}{}{}
\cventry{2017 -- }{Scientific Advisory Board}{NSF NeuroNex}{Enhanced resolution for 3DEM analysis of synapses across brain regions and taxa. Provided scientific, computational, and statistical guidance to a flagship NSF funded BRAIN Initiative program}{}{}
\cventry{2017 -- }{Chair of Committee of Data  Cores}{U19 Data Cores}{}{The U19 program is NIH's flagship BRAIN Initative program, with five original awardees, each with a dedicated Data  Core and designated PI. I was elected the chair of the committee of Data Core PIs}{}
\cventry{2017}{Consultant for Nature Publishing Group}{}{}{The journal Nature, flagship journal of Nature Publishing Group, decided to create a ``Code and Software Submission Checklist''.  They consulted me on their first draft, and I helped re-write it. An image of the final checklist is available \href{https://github.com/jovo/cv/raw/master/Code-and-Software-Submission-Checklist.png}{here}}{}
\cventry{2011 -- }{Open Connectome Project}{}{}{The co-founder of the ``Open Connectome Project'' (OCP), for several years, I was the only neuroscientist that could easily store, manage, and analyze very big datasets, spanning first tens of terabytes, and then hundreds.  For that reason, I was an essential co-author on a number of big data papers. Specifically, though I sometimes contributed relatively little to the scientific ideas, I often was required to complete, visualize, and/or share the data. Perhaps more importantly, both funding agencies and journals began mandating that these large datasets be publicly shared, and OCP was literally the only option. This is despite often not having funding, nor being a co-author, on the manuscripts}{}
\cventry{2010 -- }{AWS Open Neuro Data Registry}{}{}{Our lab co-founded the \href{https://registry.opendata.aws/open-neurodata/}{Registry of Open Data on Amazon Web Serivces} (AWS). The implication of this is that now, pending a few minor considerations, any neuroscientist that collects large image data can deposit it online \textit{for free}.  This means that neither they nor we must request funding to store the data. Our lab maintains this repository, but only by virtue of ensuring instructions for uploading, visualizing, and downloading are up to date, and acting as a gatekeeper to ensure only appropriate data are deposited there}{}


\subsection{Journal Service}

\subsubsection{Editorial Board}
    \cventry{2019 -- }{Associate Editor}{Journal of the American Statistical Association}{}{}{}
    \cventry{2018 -- }{Editor}{Neurons, Behavior, Data analysis, and Theory}{}{}{}
    \cventry{2016}{Guest Associate Editor}{PLoS Computational Biology}{}{}{}

\subsubsection{Conference and Journal Reviewer}
    \cventryreviewer{}{Annals of Applied Statistics (AOAS)}{}{}{}{}
    \cventryreviewer{}{Bioinformatics}{}{}{}{}
    \cventry {} {International Conference on Learning Representations (ICLR)} {}{}{}{}
    \cventry {} {Network Science} {}{}{}{}
    \cventry {} {Current Opinion in Neurobiology} {}{}{}{}
    \cventry {} {Biophysical Journal} {}{}{}{}
    \cventry {} {IEEE International Conference on eScience} {}{}{}{}
    \cventry {} {IEEE International Conference on Acoustics, Speech, and Signal Processing (ICASSP)}{}{}{}{}
    \cventry{}{IEEE Global Conference on Signal and Information Processing (GlobalSIP)}{}{}{}{}
    \cventry {} {IEEE Signal Processing Letters} {} {} {} {}
    \cventry {} {IEEE Transactions on Signal Processing} {}{}{}{}
    \cventry {} {Frontiers in Brain Imaging Methods} {}{}{}{}
    \cventry {} {Journal of Machine Learning Research (JMLR)} {}{}{}{}
    \cventry {} {Journal of Neurophysiology} {}{}{}{}
    \cventry {} {Journal of the Royal Statistical Society B (JRSSB)} {}{}{}{}
    \cventry {} {Nature Communications} {}{}{}{}
    \cventry {} {Nature Methods} {}{}{}{}
    \cventry {} {Nature Reviews Neuroscience} {}{}{}{}
    \cventry {} {Neural Computation} {}{}{}{}
    \cventry {} {Neural Information Processing Systems (Neurips)} {}{}{}{}
    \cventry {} {NeuroImage} {}{}{}{}
    \cventry {} {Neuroinformatics} {}{}{}{}
    \cventry {} {PLoS One} {}{}{}{}
    \cventry {} {PLoS Computational Biology} {}{}{}{}


\subsection{Conferences and Hackathon Organizer}
\cventry{Summer '20}{Co-Chair}{SciPy mini-symposium: Biology and Bioinformatics}{}{}{}{}
\cventry{Winter '19}{Organizer}{Decision Forest Hackathon}{}{}{}
\cventry{Summer '19}{Organizer}{NeuroData Workshop}{\url{https://neurodata.devpost.com}}{Hackashop to train brain scientists in machine learning for big data ($\sim$ 50) participants from around the country}{}
\cventry{March '19}{Organizer}{Neuro Reproducibility Hackashop}{\url{https://brainx3.io/}}{Hackashop to train brain scientists in best practices in reproducible science, co-organized with two startups: Vathes, LLC and Gigantum ($\sim$ 50 participants)}{}
\cventry{Spring '18}{Organizer}{NeuroData Hackathon}{}{}{}    
\cventry{Fall '17}{Organizer}{NeuroData Mini-Hackathon}{}{}{}
\cventry{Summer '17}{Organizer}{NeuroStorm}{\url{https://brainx2.io}}{Workshop to bring together thought leaders from academia, national labs, industry, and non-profits around the world to take next steps towards accelerating brain science discovery in the cloud ($\sim$ 50 participants and 5 observers from funding institutions)}{}
\cventry{2016}{Organizer}{Global Brain Workshop}{\url{http://brainx.io}}{First ever international Brain Initiative workshop, bringing together leaders from around the world, covered by Nature and Science ($\sim$ 75 participants)}{}    
\cventry{2016}{Co-Organizer}{Brains and Bits: Neuroscience Meets Machine Learning, NIPS Workshop}{\url{http://www.stat.ucla.edu/~akfletcher/brainsbits_overview.html}}{}{}
\cventry{Winter '15}{Organizer}{Hack@NeuroData}{\url{http://hack.neurodata.io/}}{}{}
\cventry{2015}{Co-Organizer}{BigNeuro2015: Making Sense of Big Neural Data, NIPS Workshop}{\url{http://neurodata.io/bigneuro2015}}{}{}    
\cventry{2012}{Co-Organizer}{\href{https://openwiki.janelia.org/wiki/download/attachments/8687459/final+agenda+EM+Connectomics+100512.pdf}{Scaling up EM Connectomics Conference}}{}{The world's first connectomics workshop, now run annually alternating between Janelia Research and Max Plank locations ($\sim$ 80 participants)}{}

\section{\href{https://neurodata.io/about/awards/}{Awards and Recognition}}
\subsection{Individual}
    \cventry{2002}{Dean's List}{Washington University}{}{}{}
    
\subsection{Shared (10)}
    \cventry{2019}{\href{https://kavlijhu.org/funding/awards}{Kavli NDI Distinguished Postdoctoral Fellow}}{Celine Drieu, PhD}{}{}{}{}
    \cventry{2019}{\href{https://kavlijhu.org/funding/awards}{Kavli NDI Distinguished Postdoctoral Fellow}}{Austin Graves, PhD}{}{}{}{}
    \cventry{2019}{Winner of Pistritto Fellowship.}{Vivek Gopalakrishnan}{}{}{}{}
    \cventry{2017}{\href{https://kavlijhu.org/funding/awards}{Kavli NDI Distinguished Postdoctoral Fellow}}{Audrey Branch, PhD}{}{}{}{} 
    \cventry{2017}{\href{http://www.hpdc.org/2017/awards/best-paper-award}{Best Presentation Award HPDC}}{Mhembere et al. (2017)}{}{}{}
    \cventry{2017}{Nonparametric Statistics of the American Statistical Association Student Paper Award}{Lee et al. (2017)}{}{}{}
    \cventry{2014}{F1000 Prime Recommended}{Vogelstein et al. (2014)}{}{}{}
    \cventry{2013}{Spotlight}{Neural Information Processing Systems (NIPS)}{}{}{}
    \cventry{2011}{Trainee Abstract Award}{Organization for Human Brain Mapping}{}{}{}
    \cventry{2008}{Spotlight}{Computational and Systems Neuroscience (CoSyNe)}{}{}{}

\subsection{Other Media}
    \cventry{2020}{A. Mandavilli}{\href{https://www.nytimes.com/2020/06/11/health/coronavirus-cytokine-storm.html}{Coronavirus Can Set Off a 'Cytokine Storm.' These Drugs May Calm It}}{NYT}{}{}{}{}
    \cventry{2020}{A. Mone}{\href{https://www.hopkinsmedicine.org/news/newsroom/news-releases/}{Researchers Urge Clinical Trial of Blood Pressure Drug to Prevent Lethal Complication of Covid-19}}{Johns Hopkins Medicine}{}{}{}{}
    \cventry{2020}{M. Rosen}{\href{https://www.hhmi.org/news/preventing-cytokine-storm-may-ease-severe-covid-19-symptoms}{Preventing 'Cytokine Storm' May Ease Severe COVID-19 Symptoms}}{HHMI}{}{}{}{}
    \cventry{2020}{J. Hamblin}{\href{https://www.theatlantic.com/health/archive/2020/04/coronavirus-immune-response/610228/}{Why Some People Get Sicker Than Others}}{The Atlantic}{}{}{}{}
    \cventry{2020}{S. Salzberg}{\href{https://www.forbes.com/sites/stevensalzberg/2020/04/08/a-possible-treatment-for-covid-19/?sh=56b3d56d75af}{Prazosin Might Be A Treatment For COVID-19. More Data Is Urgently Needed}}{Forbes}{}{}{}{}
    \cventry{2019}{Johns Hopkins Medicine}{\href{https://vimeo.com/hartlove/review/331872774/9dacde8acb}{BME Pioneers: Joshua Vogelstein}}{BME Pioneers}{}{}{}{}
    \cventry{2019}{Johns Hopkins Medicine}{\href{https://www.youtube.com/watch?v=NqmbDPzJBgI}{Technology Connecting the Brain to the Human Experience}}{}{}{}{}
    \cventry{2018}{J. M. Perkel.}{\href{https://www.nature.com/articles/d41586-018-07195-2}{Web service makes big data available to neuroscientists}}{Nature}{}{}{}{}
    \cventry{2016}{Emerging Technology from the arXiv}{\href{https://www.technologyreview.com/s/602274/three-grand-challenges-for-brain-science-that-can-be-solved-in-10-years/}{Three Grand Challenges for Brain Science That Can Be Solved in 10 Years}}{MIT}{}{}{}{}
    \cventry{2016}{P. Patel.}{\href{https://hub.jhu.edu/2016/11/02/johns-hopkins-neuroscientists-brain-trust/}{Johns Hopkins researchers want to use big data to chart the brain}}{Johns Hopkins University}{}{}{}{}
    \cventry{2016}{S. Reardon.}{\href{http://www.nature.com/news/worldwide-brain-mapping-project-sparks-excitement-and-concern-1.20658}{Worldwide brain-mapping project sparks excitement --- an concern}}{Nature}{}{}{}{}
    \cventry{2016}{E. Underwood.}{\href{http://science.sciencemag.org/content/352/6283/277}{International brain projects proposed}}{Science}{}{}{}{}
    \cventry{2016}{National Institutes of Health}{\href{http://www.braininitiative.org/2016/04/22/international-brain-projects-considered/}{International Brain Projects Considered}}{BRAIN initiative}{}{}{}{}
    \cventry{2016}{Office of the Spokesperson}{\href{https://2009-2017.state.gov/r/pa/prs/ps/2016/09/262200.htm}{International Brain Initiative Launch and VIP Dialog: Towards an International Brain Station}}{US Department of State}{}{}{}{}
    \cventry{2016}{The Kavli Foundation}{\href{http://www.kavlifoundation.org/international-brain-initiative}{International Brain Initiative}}{Kavli}{}{}{}{}
    \cventry{2015}{Dale Keiger}{\href{https://hub.jhu.edu/magazine/2015/winter/open-connectome-project-joshua-vogelstein-randal-burns/}{The Open Connectome Project takes a close look at the brain}}{Johns Hopkins Magazine}{}{}{}{}
    \cventry{2014}{S. Begley}{\href{http://www.reuters.com/article/us-science-flybrain/fly-brain-atlas-opens-door-to-linking-human-neurons-to-actions-idUSBREA2Q23B20140327}{Fly brain 'atlas' opens door to linking human neurons to actions}}{Reuters}{}{}{}{}
    \cventry{2014}{L. Gatlin}{\href{https://hub.jhu.edu/2014/08/22/cary-priebe-nsf-eager-grant/}{Johns Hopkins mathematician receives grant to support study of brain's circuitry}}{Johns Hopkins University}{}{}{}{}
    \cventry{2014}{T. O'Leary and E. Marder}{\href{http://science.sciencemag.org/content/344/6182/372.full}{Mapping Neural Activation onto Behavior in an Entire Animal}}{Science}{}{}{}{}
    \cventry{2014}{L. Sanders}{\href{https://www.sciencenews.org/node/188288}{Ten thousand neurons linked to behaviors in fly}}{ScienceNews}{}{}{}{}
    \cventry{2014}{D. Son and J. Lee}{\href{http://www.cis.jhu.edu/~parky/CEP-Publications/Nature-highlight.pdf}{Research Highlights}}{Nature}{}{}{}{}
    \cventry{2014}{K. Yandell}{\href{http://www.the-scientist.com/?articles.view/articleNo/39571/title/Linking-Neurons-to-Behaviors}{Linking Neurons to Behaviors}}{TheScientist}{}{}{}{}
    \cventry{2014}{B. Yirka}{\href{https://medicalxpress.com/news/2014-03-atlas-neural-circuits-fruit-larvae.html}{Researchers create a reference atlas for neural circuits in fruit fly larvae}}{MedicalXpress}{}{}{}{}
    \cventry{2012}{C. B. Begg and M. C. Pike}{\href{http://stm.sciencemag.org/content/4/135/135le3.full}{Comment on "The Predictive Capacity of Personal Genome Sequencing"}}{Science}{}{}{}{}
    \cventry{2012}{B. Thomas}{\href{https://theconnectome.wordpress.com/2012/02/05/podcast-1-our-interview-with-joshua-vogelstein/}{“Open Access to the Brain” – Podcast 1: Joshua Vogelstein}}{The Connectome Podcast}{}{}{}{}
    \cventry{2012}{E. J. Topol}{\href{http://stm.sciencemag.org/content/4/135/135le5.full}{Comment on "The Predictive Capacity of Personal Genome Sequencing"}}{Science}{}{}{}{}

\subsection{Professional/Social Media Presence}
    % \todo{JOVO: put the $\sim$30 blogs here.}
    \cventry{}{\href{https://twitter.com/neuro_data}{@neuro\_data}}{}{Twitter account with a approximately 7,000 followers, over 250K impressions in December 2019, and approximately 100 new followers, and upwards of 100 new tweets, per month, and 25 link clicks per day. Follower demographics include $< 50\%$ high school graduates, $46\%$ female}{}{}{}
    \cventry{}{\href{https://bitsandbrains.io/}{Bits and Brains}}{}{Professional blog reguarding all things academic, neurological, and statistical, with approximately 30 blog posts, approximately one new post per month (9,000 page views, 3,200 unique users) \newline Most Popular Post: \href{https://bitsandbrains.io/2019/02/10/how-to-write-a-paper.html}{10 Simple Rules to Write a Paper from Start to Finish}}{}{}{}

\section{Translation / Technology Transfer Activities}
    \subsection{Open Datasets}
        \cventry{2019 -- }{\href{https://neurodata.io/data/templier2019}{Templier et al.  (2019)}}{}{The non-destructive collection of ultrathin sections onto silicon wafers for post-embedding staining and volumetric correlative light and electron microscopy using MagC. MagC allows the correlative visualization of neuroanatomical tracers within their ultrastructural volumetric electron microscopy context}{}{0 citations, 119 unique visitors}{}
        \cventry{2018 -- }{\href{https://neurodata.io/data/bloss2018}{Bloss et al. (2018)}}{}{Images of CA1 pyramidal neurons for analysis involving feature-selective firing as a result of dendritic integration of inputs from multiple brain regions. Show that single presynaptic axons form multiple, spatially clustered inputs onto the distal, but not proximal, dendrites of CA1 pyramidal neurons}{}{20 citations, 530 unique visitors}{}
        \cventry{2018 -- }{\href{https://neurodata.io/data/branch18}{Branch (2018)}}{}{Adult generated neurons in aging M. musculus imaged using array tomography, multi-spectral light microscopy, and  electron microscopy}{}{2 citations, 223 unique visitors}{}
        \cventry{2017 -- }{\href{https://neurodata.io/data/allen_atlas}{Allen Atlas}}{}{Anatomical reference atlases that illustrate the adult mouse brain in coronal and sagittal planes. They are the spatial framework for datasets such as in situ hybridization, cell projection maps, and in vitro cell characterization.\href{http://atlas.brain-map.org/}{atlas.brain-map.org}}{}{142 citations, 1058 unique visitors}{}
        \cventry{2017 -- }{\href{https://neurodata.io/data/hildebrand17}{Hildebrand et al. (2017)}}{}{A multi-resolution serial-section electron microscopy data set containing the anterior quarter of a 5.5 days post fertilization larval zebrafish, including its complete brain acquired by Hildebrand and colleagues. Electron micrographs and reconstructions are available for view in CATMAID}{}{70 citations, 1,014 unique visitors}{}
        \cventry{2017 -- }{\href{https://neurodata.io/data/tobin17}{Tobin et al. (2017)}}{}{Wiring variations that enable and constrain neural computation in a sensory microcircuit}{}{28 citations, 43 unique visitors}{}
        \cventry{2016 -- }{\href{https://neurodata.io/data/bloss2016}{Bloss et al. (2016)}}{}{Images of molecularly defined inhibitory interneurons and CA1 pyramidal cell dendrites collected using correlative light-electron microscopy and large-volume array tomography}{}{41 citations, 701 unique visitors}{}
        \cventry{2016 -- }{\href{https://neurodata.io/data/xbrain}{Dyer et al.  (2016)}}{}{Mesoscale (1 cubic micron resolution) resolution images generated with the use of synchrotron X-ray microtomography (microCT) from millimeter-scale volumes of mouse brain. X-ray tomography promises rapid quantification of large brain volumes}{}{21 citations, 216 unique visitors}{}
        \cventry{2016 -- }{\href{https://neurodata.io/data/lee16}{Lee et al. (2016)}}{}{Electron microscopy data collected at $4 \times 4 \times 40$ nm per voxel from the visual cortex in Mouse V1 used in a study of an excitatory network}{}{132 citations, 725 unique visitors}{}
        \cventry{2016 -- }{\href{https://neurodata.io/data/wanner16}{Wanner et al. (2016}}{}{Serial block face scanning EM (SBEM) and conductive sample embedding image stack from an olfactory bulb (OB) of a zebrafish larva at a voxel resolution of $9.25 \times 9.25 \times 25$ nm3}{}{12 citations, 328 unique visitors}{}
        \cventry{2015 -- }{\href{https://neurodata.io/data/bigbrain}{Amunts et al. (2015)}}{}{BigBrain is an ultrahigh-resolution three-dimensional model of a full human brain at 20 micrometer resolution, enabling an unprecedented look into the human brain at micro- and macro-scopic scale}{}{262 citations, 1,041 unique visitors}{}
        \cventry{2015 -- }{\href{https://neurodata.io/data/bhatla15}{Bhatla et al. (2015)}}{}{Nikhil Bhatla and Rita Droste in Bob Horvitz's Lab reconstruction of the anterior half of the C. elegans feeding organ, the pharynx. Volumes for three adult hermaphrodite worms include volumetric tracing of all neurons, selected cell types, I2 neuron synapses. 50 nm thick sections with an image resolution of 2 nm per pixel}{}{16 citations, 467 unique visitors}{}
        \cventry{2015 -- }{\href{https://neurodata.io/data/collman15}{Collman et al. (2015)}}{}{Mouse cortex collected using conjugate array tomography (AT), a volumetric imaging method that integrates immunofluorescence and EM imaging modalities in voxel-conjugate fashion}{}{69 citations, 382 unique visitors}{}
        \cventry{2015 -- }{\href{https://neurodata.io/data/tomer15}{Deisseroth et al. (2015)}}{}{Twelve CLARITY mouse brains (5 wild type controls and 7 behaviorally challenged) were prepared by Li Ye, and imaged using CLARITY-Optimized Light-sheet Microscopy (COLM) (whole brain COLM imaging and data stitching performed by R. Tomer, in preparation)}{}{5 citations, 208 unique visitors}{}
        \cventry{2015 -- }{\href{https://neurodata.io/data/kharris15}{Harris et al.  (2015)}}{}{Three volumes of hippocampal CA1 neuropil in adult rat imaged by the laboratory of Kristen M Harris, PhD, at an XY resolution of ~2 nm on serial sections of ~50-60 nm thickness}{}{9 citations, 463 unique visitors}{}
        \cventry{2015 -- }{\href{https://neurodata.io/data/kasthuri15}{Kasthuri et al. (2015)}}{}{Saturated reconstruction of a sub-volume of mouse neocortex collected using automated technologies in which all cellular objects (axons, dendrites, and glia) and many sub-cellular components are rendered and itemized in a database. Provides access to the complexity of the neocortex and enables further data-driven inquiries}{}{323 citations, 1,299 unique visitors}{}
        \cventry{2015 -- }{\href{https://neurodata.io/data/kristina15}{Micheva et al. (2015)}}{}{Multi-channel array tomography data of the barrel cortex of an adult mouse (C57BL/6J)}{}{57 citations, 190 unique visitors}{}
        \cventry{2015 -- }{\href{https://neurodata.io/data/acardona_0111_8}{Ohyama et al. (2015)}}{}{The side view of the approximately 7,000 neurons reconstructed so far, either in full or partially, of the approximately 12,000 neurons of the central nervous system of Drosophila larva. The 0111-8 data set was originally sectioned and imaged by Richard D. Fetter and his two tech assistants}{}{136 citations, 299 unique visitors}{}
        \cventry{2015 -- }{\href{https://neurodata.io/data/zbrain_atlas}{Randlett et al. (2015)}}{}{Zebrafish brain atlas with surface mesh of different regions intended for the analysis of whole-brain activity mapping}{}{124 citations, 498 unique visitors}{}
        \cventry{2014 -- }{\href{https://neurodata.io/data/weiler14}{Weiler (2014)}}{}{Images of whisker-associated barrel columns of mouse somatosensory cortex stained with antibodies against selected antigens (DAPI, YFP), and indirect immunofluorescence. Images collected by the lab of Stephen J Smith}{}{6 citations, 123 unique visitors}{}
        \cventry{2013 -- }{\href{https://neurodata.io/data/bumbarger13}{Bumbarger et al. (2013)}}{}{Serial, thin section data generated by Dan Bumbarger in Ralf Sommer's lab in order to compare the pharyngeal connectomes of the pharyngeal nervous system between Caenorhabditis elegans and Pristionchus pacificus. In P. pacificus they found clearly homologous neurons for all of the 20 pharyngeal neurons in C. elegans, and massive rewiring of synaptic connectivity between the two species}{}{67 citations, 22 unique visitors}{}
        \cventry{2013 -- }{\href{https://neurodata.io/data/takemura13}{Takemura et al. (2013)}}{}{The right part of the brain of a wild-type Oregon R female fly that was serially sectioned into 40-nm slices. A total of 1,769 sections, traversing the medulla and downstream neuropils, were imaged at a magnification of 35,000X}{}{323 citations, 144 unique visitors}{}
        \cventry{2011 -- }{\href{https://neurodata.io/data/bock11}{Bock et al. (2011)}}{}{Volume of mouse primary visual cortical data, spanning layers 1, 2/3, and upper layer 4 collected as electron microscope (EM) data and two-photon microscopy data collected by Davi Bock, Ph.D. and Wei-Chung Allen Lee, Ph.D.. Images have a resolution of 4x4x45 cubic nanometers}{}{430 citations, 511 unique visitors}{}
    
    \subsection{Open-Source Software: Active}
     
        Stars denote an individual users appreciation, downloads indicates a user downloading the code, and a fork indicates a user modifying the code.
        
            \cventry{2020 -- }{\href{https://github.com/neurodata/ProgLearn}{ProgLearn (Progressive Learning)}}{}{A Python package for exploring and using progressive learning algorithms}{}{22 stars, 29 forks, 37 downloads/month}{}
            \cventry{2019 -- }{\href{https://github.com/neurodata/ardent}{ARDENT (Affine and Regularized Deformative Numeric Transform)}}{}{A Python package for performing automated image registration using LDDMM}{}{10 stars, 5 forks}{}
            \cventry{2019 -- }{\href{https://neurodata.io/graspy/}{graspologic (Graph Statistics)}}{}{Co-developed with Microsoft Research: Utilities and algorithms designed for processing and analysis of graphs with specialized graph statistical algorithms}{}{134 stars, 56 forks, 2,516 downloads/month}{}
            \cventry{2019 -- }{\href{https://neurodata.io/reg/}{reg (Image registration)}}{}{A Python package which performs non-linear affine and deformable image registration}{}{6 stars, 4 forks, 61 downloads/month}{}
            \cventry{2019 -- }{\href{https://github.com/neurodata/neuroparc}{neuroparc}}{}{This repository contains a number of useful parcellations, templates, masks, and transforms to (and from) MNI152NLin6 space. The files are named according to the BIDs specification}{}{26 stars, 4 forks}{}
            \cventry{2019 -- }{\href{https://neurodata.io/sporf/}{Sparse Projection Oblique Randomer Forests (Classification and regression)}}{}{SPORF is an improved random forest algorithm that achieves better accuracy and scaling than previous implementations on a standard suite of > 100 benchmark problems}{}{54 stars, 35 forks, 73 downloads/month, 36 docker pulls}{}
            \cventry{2019 -- }{\href{https://github.com/neurodata/uncertainty-forest}{Uncertainty-Forest}}{}{A Python package containing estimation procedures for posterior distributions, conditional entropy, and mutual information between random variables X and Y}{}{2 stars, 1 fork}{}
            \cventry{2018 -- }{\href{https://neurodata.io/lol/}{LOL (Supervised dimensionality reduction)}}{}{Linear Optimal Low-rank (LOL) projection for improved classification performance in high-dimensional classification tasks}{}{8 stars, 6 forks, 60 downloads/month}{}
            \cventry{2018 -- }{\href{https://neurodata.io/mgc/}{MGC (Non-parametric hypothesis testing)}}{}{Multiscale Graph Correlation (MGC) is a framework for universally consistent testing high-dimensional and non-Euclidean data}{}{28 stars, 11 forks, 120 downloads/month, 266 docker pulls}{}
            \cventry{2018 -- }{\href{https://neurodata.io/m2g/}{m2g (MR graph analysis)}}{}{A Python pipeline which uses diffusion MRI data from individuals to generate connectomes reliably and scalably}{}{35 stars, 26 forks, 218 downloads/month, 7,900 docker pulls}{}
            \cventry{2018 -- }{\href{https://neurodata.io/nd_cloud/}{ndcloud (NeuroData Cloud)}}{}{The deployment of tools which support the Open Connectome Project}{}{}{}
            \cventry{2016 -- }{\href{https://github.com/neurodata/non-parametric-clustering}{Non-Parametric-Clustering}}{}{A program which uses non-parametric-clustering to minimize or maximize a given criterion function}{}{3 stars, 2 forks}{}
        
    \subsection{Open-source Software: Contributed}
            \cventry{2019}{\href{https://github.com/neurodata/cloud-volume}{cloud-volume}}{}{Added support for additional file types}{}{}{}
            \cventry{2019}{\href{https://github.com/FCP-INDI/C-PAC}{C-PAC}}{}{Added streamlined reproducible pipeline}{}{}{}
            \cventry{2019}{\href{https://github.com/neurodata/scipy}{scipy}}{}{Added mgc, a state of the art method for hypothesis testing we developed in the lab}{}{}{}
            \cventry{2018 -- 2019}{\href{https://github.com/neurodata/neuroglancer}{neuroglancer}}{}{Added multispectral support to enable light microscopy data use}{}{}{}
            \cventry{2018}{\href{https://igraph.org}{igraph}}{}{Added spectral clustering functionality}{}{}{}
            \cventry{2017 -- 2018}{\href{https://github.com/neurodata/render}{render}}{}{Added cloud support}{}{}{}
            \cventry{2017}{\href{https://github.com/neurodata/boss}{boss}}{}{Developed core functionality}{}{}{}
        
        
    \subsection{Open-source Software: Archived}
            \cventry{2017 -- 2019}{\href{https://github.com/neurodata/ndex}{ndex}}{}{Python 3 command-line program to exchange (download/upload) image data with NeuroData's cloud deployment of APL's BOSS spatial database}{}{3 stars, 0 forks, 89 downloads/month}{}
            \cventry{2017 -- 2019}{\href{https://github.com/flashxio/knorPy}{knor (Clustering)}}{}{Python version of knor, a highly optimized and fast library for computing k-means in parallel with accelerations for Non-Uniform Memory Access (NUMA) architectures}{}{1 stars, 3 forks, 115 downloads/month}{}
            \cventry{2017 -- 2019}{\href{https://github.com/aksimhal/SynapseAnalysis}{SynapseAnalysis (Synapse Detection)}}{}{A framework to evaluate synaptic antibodies for array tomography applications}{}{2 stars, 0 forks}{}
            \cventry{2017 -- 2018}{\href{https://github.com/neurodata/pymeda}{MEDA (Matrix Exploratory Data Analysis)}}{}{A python package for matrix exploratory data analysis}{}{0 stars, 3 forks, 56 downloads/month, 21 docker pulls}{}
            \cventry{2017 -- 2018}{\href{https://github.com/neurodata/ndwebtools}{ndwebtools}}{}{ndwebtools (ndweb) is a Django application to provide a user-friendly interface for interacting with NeuroData resources and data}{}{0 stars, 1 forks}{}
            \cventry{2015 -- 2018}{\href{https://github.com/neurodata/ndviz}{ndviz}}{}{Web visualization and analysis tools for neuroimaging datasets, powered by Neuroglancer}{}{8 stars, 4 forks, 48 docker pulls}{}
            \cventry{2015 -- 2016}{\href{https://github.com/mkazhdan/DMG}{DMG}}{}{An implementation of a distributed multigrid Poisson solver for image stitching, smoothing, and sharpenting}{}{19 stars, 6 forks}{}
            \cventry{2015}{\href{https://github.com/neurodata/vesicle}{VESICLE (EM Synapse Detection)}}{}{Reference synapse detection program for processing serial electron microscopy data}{}{3 stars, 3 forks}{}
            \cventry{2015}{\href{https://github.com/neurodata/CAJAL}{CAJAL}}{}{A MATLAB API that provides a simple to use interface with Open Connectome Project servers and provides RAMON Objects, unit tests, configuration scripts, and utilities}{}{6 stars, 5 forks}{}
            \cventry{2012 -- 2017}{\href{https://github.com/flashxio/FlashX}{FlashGraph (Scalable Analytics)}}{}{General-purpose graph analysis framework that exposes vertex-centric programming interface for users to express varieties of graph algorithms}{}{220 stars, 42 forks}{}
            \cventry{2012 -- 2017}{\href{https://github.com/flashxio/FlashX}{FlashX (Scalable machine learning)}}{}{A matrix computation engine that provides a small set of generalized matrix operations on sparse matrices and dense matrices to express varieties of data mining and machine learning algorithms}{}{220 stars, 42 forks}{}
            \cventry{2011 -- 2016}{\href{https://github.com/jovo/oopsi}{oopsi (Calcium Spike Sorting)}}{}{Model-based spike train inference from calcium imaging}{}{20 stars, 9 forks}{}
            \cventry{2011 -- 2017}{\href{https://github.com/neurodata/ndstore}{ndstore}}{}{Scalable database cluster for the spatial analysis and annotation of high-throughput brain imaging data}{}{37 stars, 13 forks}{}
        
        
        




    \subsection{Consultancy}
    \cventry{2017}{Consultant}{\href{https://www.greenspringassociates.com}{Greenspring Associates}}{}{}{}
    \cventry{2016}{Consultant}{Scanadu}{}{}{}
    




\subsection{Advisory Board Appointments}
\cventry{2018 -- }{Advisory Board}{\href{https://mind-x.io/}{Mind-X}}{A  neurotechnology company combining brain-computer interfaces and artificial intelligence to make the world’s information available with the speed and ease of a single thought.}{Incubated at Camden Partners Nexus, completed an initial round of funding for an undisclosed amount}{15  employees.}
\cventry{2017 -- }{Advisory Board}{\href{https://www.pivotalpath.com/}{PivotalPath}}{PivotalPath is a leading hedge fund research and intelligence organization built by a team of experienced alternative investment professionals and fintech developers.}{Raised undisclosed amount of funding}{11 employees.}



\subsection{Startups}
\cventry{2017 -- }{Co-Founder}{\href{http://gigantum.io}{gigantum}}{The future of data science is open, decentralized and user friendly. That is why we created a platform that enables anybody to create and share totally reproducible computational work with the world.}{Completed initial round of seed funding for undisclosed amount from \href{https://www.digital-science.com/}{Digital Science}, which also funds figshare, readcube, altmetric, overleaf, and more}{15 employees.}
\cventry{2016 -- }{Co-Founder}{\href{http://www.d8alab.com}{d8alab}}{Our services include evaluating model performance, building prototype R/Shiny web applications and basic data cleaning.}{Provides data science consulting for a variety of companies, specifically biomedical data science}{4 employees.}



\end{document}
